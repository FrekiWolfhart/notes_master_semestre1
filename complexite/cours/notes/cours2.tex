\section{Cours 2}
\subsection{Mise en relation complexité et efficacité pratique}
L'objectif de cette UE est d'étudier les problèmes conjecturés de complexité exponentielle.

On travaille sur des graphes orientés finis, avec un nombre d'arcs finis.

$\Gamma^+$(x) est l'ensemble des successeurs de x.\\
$\Gamma^-$(x) est l'ensemble des prédécesseurs de x.\\
$\Gamma$(x) est l'ensemble des voisins de x.

Le graphe réciproque d'un graphe orienté est le même graphe avec la direction des arcs inversés.
Un graphe symétrisé est l'union d'un graphe orienté et de son réciproque.

\subsection{Représentation machione}
\subsubsection{Par matrice d'adjacence}
Le cas standrad est une matrice de booléen, qui est symétrique si le graphe est non-orienté.
On mets un 1 dans la case si il y a un arc entre ligne et colonne, et un 0 sinon.

Un graphe valué sera représenté par une matrice de réels, avec la valeur d'un arc dans la case correspondante, ou -1 si il n'y a pas d'arc.
\subsubsection{Par liste d'adjacence}
L'ordre dans la liste n'a pas d'importance. Il y a autant d'éléments dnas la liste qu'il y a d'arcs dans le graphe.
