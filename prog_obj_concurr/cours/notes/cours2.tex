\section{Cours 2}
\subsection{L'indépendance}
La notion d'indépendance a été théorisée en 1966.
Lorsqu'on a deux parties du programmes, on sépare chaque partie en deux parties:
\begin{itemize}
	\item les variables lues par la partie
	\item les variables écrites par la partie
\end{itemize}
Pour que les parties soient indépendantes, il faut que:
\begin{itemize}
	\item la partie 1 ne lise aucune variable écrite par la partie 2
	\item la partie 2 ne lise aucune variable écrite par la partie 1
\end{itemize}
Donc, que les deux parties ne touchent pas des variables similaires.

\subsection{La notion d'atomicité}
Dans ce cours, une instruction est dite atomique si et seulement si les variables lues ou modifiées par cette
instruction ne peuvent pas être lues ou modifiées par le reste du programme durant son exécution.
Avoir des instruction atomiques n'oblige pas le reste des threads à ne rien faire dans le programme.

Il y a deux types d'atomicité:
\begin{itemize}
	\item les instructions atomiques de base du langage
	\item la création d'instruction atomiques avec un verrou rajouté par le codeur
\end{itemize}
En java, les écritures de références sont toujours atomiques.

Dans le cas ou on mets plusieurs verrous, afin d'éviter les inter-blocages, il faut décider un ordre des verrous
dans le programme, et ne jamais le changer.

\subsection{Les variables atomiques}
Si la variable est déclaré comme atomique, AtomicInteger au lieu d'Integer, par exemple, cela permet à la variable
d'avoir les avantages de l'atomicité, sans devoir créer de verrour dans le code.
