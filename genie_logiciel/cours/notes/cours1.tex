\section{Cours 1}
Un mauvais développement peux créer des erreures graves,\\
mais même le plus parfait des développements ne peux être exempt d'erreurs.

Les principales sources d'erreurs sont:
\begin{itemize}
	\item Les erreurs humaines,
	\item Les problèmes trop complexes,
	\item Les demandes du client, tel que le rajout de fonctionnalités en plein milieu du projet.
\end{itemize}

\subsection{Qualités logicielles}
\subsubsection{Les qualités externes}
Les qualités externes du logiciel sont des qualités remarquables par tout le monde, même les non programmeurs.
Par exemple:
\begin{itemize}
	\item validité, n'importe qui peux remarquer, et même des fois assez vite, si le code n'est pas valide,
	car il ne répondra donc pas aux besoins.
	\item robustesse, si le code n'est pas robuste, et qu'il casse, les utilisateurs s'en rendront compte
	facilement.
	\item performance, un programme qui lag n'est pas vraiment génial à utiliser.
	\item ergonomie, si l'interface n'est aps ergonomique, cela va ruiner l'expérience de l'utilisateur.
\end{itemize}
\subsubsection{Les qualités internes}
Les qualités internes sont des qualités de code remarquées principalement par des développeurs.
Par exemple:
\begin{itemize}
	\item modulabilité, si le code n'est pas miodulable, rajouter de nouvelles fonctionnalités sera une horreur.
	\item lisibilité, le debuggage sera presque impossible si le code est illisible.
	\item maintenabilité, si on ne peux aps facilement débugger le code, il va souvent crasher, et les
	utilisateurs ne seront pas content.
\end{itemize}

Tout ces facteurs n'étant pas forcément compatibles entre eux, il faudra compromiser, voir en prioriser certains.

Quand on design un projet, on doit satisfaire les critères CQFD, mais les facteurs coût et délai de développement
peuvent être limitants.

\subsection{Le cycle de vie}

La gestion du projet est une part importante du développement logiciel, car elle permet de bien
séparer les tâches, et d'organiser le projet.

\subsubsection{Exemples de modèles de gestion}
Le modèle en V a été conçu pour corriger la lacune principale du modèle en cascade, qui est le test en
fin de projet.

Dans le modèle en V, chaque composante a ses tests et sa réalisation fait en parallèle, ce qui permet
d'identifier plus facilement les sources d'erreurs.

Le modèle en spirale est surtout utilisé pour les projets dits innovants, c'est à dire les projets où le client
ne sait pas précisément ce qu'il veut. Il consiste à présenter de multiples prototypes au client, et à ne
continuer que avec ceux approuvés, ce jusqu'à réalisation du produit final.

\subsection{Méthodes et modèles}

Un bon modèle se doit d'être abstrait, raffiné et lisible.\\
Les modèles les plus utilisés en informatique sont les modèles conceptuels, qui se basent sur l'utilisation
de symboles pour représenter des aspects qualitatifs.
