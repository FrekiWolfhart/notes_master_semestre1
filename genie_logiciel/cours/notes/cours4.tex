\section{Cours 4}

\subsection{Diagramme de classe}
La classe est représentée par un rectangle qui contient le nom.
Il peux aussi contenir des méthodes et variables, mais cela est inutile pour la phase de spécification.

Il est possible tracer un trait entre deux classes pour les associer.
Sur le trait, on peut mettre des valeurs de cette forme:
\begin{itemize}
	\item rien, un seul
	\item * de 0 à +$\infty$
	\item 0..n, de 0 à n
\end{itemize}
Il est possible, et conseillé, de marquer sous la valeur ce que ça représente.
On peux aussi marquer sur le trait l'action que fait la valeur.

L'agrégation indique que l'objet duquel par la flèche est identifiable par l'objet qui reçoit la flèche, mais
l'inverse n'est pas vrai.

Dans la composition, la destruction de l'élement qui reçoit la flèche détruit les élements desquels partent la
flèche. L'inverse n'est pas vrai.
