\section{Cours 2}
\subsection{Méthodes et modèles(suite)}

On doit d'abord modéliser l'environnement du projet, c'est à dire les objets et leurs états.

En utilisant des langages comme l'UML, on peux créer une machine à états qui va décrire le comportement
des objets en fonction des commandes qui leur sont envoyées.

On peux imaginer une machine à états comme un automate, dont les commandes sont les transition,
et dont l'état par défault est l'état initial.\\
\begin{figure}[ht]
	\centering
	\begin{tikzpicture}
		\node[state, initial] at(0,0) (0){fermée};
		\node[state]          at(6,0) (1){ouverte};
		\draw[->] (0) edge[loop above] node{fermer} (0)
				  (1) edge[loop above] node{ouvrir} (1)
				  (0) edge[above]      node{ouvrir} (1)
				  (1) edge[below]      node{fermer} (0);
	\end{tikzpicture}
	\caption{Exemple de modélisation d'une trappe.}
\end{figure}

Si il y a plus d'une donnée, l'état devra être décrit par l'ensemble des données, ce qui est plus complexe
que dans l'exemple de la figure 1.

Il faut aussi, à travers le modèle, prévoir les cas où il ne se passe rien. Per exemple, si un chariot se trouve à
sa position la plus à gauche, et qu'on lui demande d'aller à gauche, il doit rester immobile.
\begin{figure}[ht]
	\centering
	\begin{tikzpicture}
		\node[state, initial] at(0,0)  (0){$P \leq P_2$};
		\node[state]          at(4,2)  (1)[text width=3cm]{déplacement à droite};
		\node[state]          at(4,-4) (2)[text width=3cm]{déplacement à gauche};
		\node[state]          at(8,0)  (3){arrêt $P>P_2$};
		\draw[->] (0) edge[above] node{+v}                              (1)
				  (1) edge[above] node{$\emptyset$}                     (3)
				  (3) edge[bend right, above] node{-v}                  (2)
				  (2) edge[below] node{$\emptyset (P \leq P_2)$}        (0)
				  (2) edge[bend right, above] node{$\emptyset (P>P_2)$} (3);
	\end{tikzpicture}
	\caption{Modèle du déplacement du chariot.}
\end{figure}
\newpage

\subsubsection{Système de contrôle}
Le système de contrôle est me système qui lie tout les objets dans un environnement.

L'opérateur (humain) ordonne le démarrage, ou l'arrêt d'urgence, du système, et le système ordonne aux objets d'agir
en conséquence.

Le système peux connaitre, i.e. stocker, des informations sur les objets qui reçoivent ses ordres.

\begin{figure}[ht]
	\centering
	\begin{tikzpicture}
		\node[state, initial] at(0,0)  (0){à l'arrêt $P_2$};
		\node[state]          at(6,0)  (1)[text width=2cm]{chariot se déplace};
		\node[state]          at(12,0) (2)[text width=2cm]{chariot à l'arrêt};
		\node[state]          at(6,-4) (3)[text width=2cm]{déplacement à gauche};
		\node[state]          at(0,-4) (4){vidage chariot};
		\draw[->] (0) edge[above] node[scale=0.5]{cycle/+v}                                         (1)
				  (1) edge[above] node[scale=0.5]{capteur droit/ $\emptyset$ mettre\_en\_marche}      (2)
				  (2) edge[right] node[scale=0.5]{temps$\geq$temps\_remplissage/mettre\_à\_l'arrêt -v}  (3)
				  (3) edge[below] node[scale=0.5]{capteur gauche/ $\emptyset$ ouvrir}               (4)
				  (4) edge[left]  node[scale=0.5]{temps$\geq$temps\_vidage/fermer}     (0);
	\end{tikzpicture}
	\caption{Modèle d'un système de contrôle.}
\end{figure}


