\section{Introduction}
\subsection{Problèmes}
L'objectif du génie logiciel est de répondre à deux problèmes principaux des projets en entreprise:
\begin{enumerate}
	\item Comment apsser du cahier des charges au code.
	\item Comment maintenir le logiciel une fois commercialisé.
\end{enumerate}

En effet, il faut du temps de réflexion pour passer du cahier des charges, décrivant comment le logiciel
doit fonctionner, à un logiciel qui correspond à ce cahier, et qui maintenable afin de ne pas avoir à
complètement le recoder à chaque changement de système ou chaque mise à jour des machines du client.

\subsection{Vocabulaire}
\begin{itemize}
	\item Le volume, cela correspond au nombre de lignes de code source du logiciel.
	\item L'effort, cela correspond au nombre d'ingénieurs mobilisé sur une unité de temps pour résoudre le
	problème.
	\item Le délai, cela correspond au délai imposé par le client ou par l'entreprise pour compléter le projet et
	le logiciel.
	\item La durée de vie, cela correspond à la durée de temps pendant laquelle le logiciel sera maintenu par
	les codeurs.
\end{itemize}

\subsection{Unités}
HA = Hommes par An.
KLS = KiloLineSource( millier de lignes de code source).
