\section{TD1}

\subsection{Exercice 1}
Un modèle de cycle de vie divisé en phases aide à la gestion du développement du logiciel, car cela permet de
\begin{itemize}
	\item bien piloter le déroulement du projet,
	\item fragmenter le projet en étapes dont l'objectif est clair,
	\item bien répartir les ressources, humaines ou matérielles,
	\item bien respecter les dépendances entre parties, et les parallèliser si possible.
\end{itemize}

\subsection{Exercice 2}
Classement:
\begin{itemize}
	\item Exigences
	\begin{enumerate}
		\item étude du marché
		\item spécification des exigences
		\item synthèse des éxigences
	\end{enumerate}
	\item Analyse
	\begin{enumerate}
		\item organisation du projet
		\item estimation des coûts
	\end{enumerate}
	\item Conception
	\begin{enumerate}
		\item conception de haut niveau
		\item conception de bas niveau
	\end{enumerate}
	\item Mise en oeuvre
	\begin{enumerate}
		\item tests unitaires
		\item implémentation
	\end{enumerate}
	\item Validation
	\begin{enumerate}
		\item tests systèmes
		\item tests d'acceptation
	\end{enumerate}
\end{itemize}

La documentation est très importante si jamais on doit changer l'équipe du projet.

\subsection{Exercice 3}
Phases:
\begin{itemize}
	\item Spécification
	\begin{itemize}
		\item Cahier des charges
		\item Plan d'assurance qualité
		\item Estimation des coûts
		\item Calendrier du projet
	\end{itemize}
	\item Conception architecturale
	\begin{itemize}
		\item Conception architecturale
		\item Spécification des modules
	\end{itemize}
	\item Conception détaillée
	\begin{itemize}
		\item Manuel utilisateur préliminaire
		\item Conception détaillée
	\end{itemize}
	\item Développement
	\begin{itemize}
		\item Code source
		\item Documentation du code
	\end{itemize}
	\item Tests unitaires
	\begin{itemize}
		\item Plan de test
	\end{itemize}
	\item Tests d'intégration
	\begin{itemize}
		\item Rapport des tests
	\end{itemize}
	\item Validation client
	\begin{itemize}
		\item Manule utilisateur final
	\end{itemize}
\end{itemize}

Le plan de test sert à vérifier que le code fait bien ce qu'il doit faire, et non pas sa qualité. C'est le plan
d'assurance qualité qui gère ça.
