\section{Cours 2}
\subsection{Couche liaisonn de données}
La couche liaison de données gère la liaison directe entre deux machines.
Dans une connection liaison de données, il n'y a pas de noeuds intermédiare entre la source et la destination.

\subsection{Couche réseau}
Elle s'occupe de l'adressage des machines, afin qu'elles puissent être identifiées à l'intérieur du réseau.

Le routage consiste à déterminer la sortie de chaque pacquet reçu par le noeud.
Le routage n'est fait que si la destination n'a pas de lien direct avec le noeud actuel.
Le routage est dit:
\begin{itemize}
	\item centralisé, si il est géré par un seul noeud pour tout le réseau,
	\item décentralisé, si chaque noeud gère le routage de ses propres paquets.
\end{itemize}

Afin de régler les problèmes de congestion, on donne un TTL(Time To Live) aux paquets qui entre le réseau.
Le TTL correspond au nombre de tronçons(liaisons) que le paquet peux emprunter avant d'atteindre sa destination.
Si le compte atteint 0 avant que la paquet n'atteigne sa destination, le paquet est détruit.
