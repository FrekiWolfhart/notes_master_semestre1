\section{Cours 1}
\subsection{Notions de base}

Un réseau n'est rien de plus qu'un graphe, c'est à dire un ensemble d'objets, ou de noeuds, connectés entre eux.

\subsubsection{Mode en diffusion}
Le mode en diffusion représente les réseaux qui transmettent les messages en les diffusant à tout les membres
du réseau. Le destinataire se doit donc d'écouter le réseau, afin de ne pas louper un message.\\
Des exemples de mode en diffusion sont:
\begin{itemize}
	\item le bus
	\item l'anneau
	\item le satellite
\end{itemize}
Sur un bus, les machines s'identifient par leurs adresses, et ne prennent le message que si l'identifiant du
destinataire du message est l'identifiant de la machine.\\
Sur un système en anneau, le message tourne dans l'anneau jusqu'à ce qu'il a atteint la machine destinataire.\\
Dans ce mode de communication, il n'y a qu'un seul message qui circule dans le réseau à la fois.

\subsubsection{Mode point à point}
Le mode point à point est le système de réseau qui ressemble le plus à un graphe, les objets sont connectés
entre eux, et font passer le message jusqu'au destinataire.\\
Des exemples de mode en point à point sont:
\begin{itemize}
	\item l'étoile
	\item la boucle simple
	\item la boucle double
	\item le maillage irrégulier
	\item le maillage régulier
\end{itemize}

Dans l'architecture en étoile, tout les membres du réseau sont reliés à un seul élément, qui sert de hub central de 
passage pour les messages. Le défaut principal de ce système est que, si le hub central tombe en panne,
plus aucune communication n'est possible.\\
Dans une boucle simple, tout les membres du réseau sont reliés à deux autres membres via un canal simple, et
font tourner le message dans le réseau jusqu'à son destinataire. Cette architecture à un point faible, c'est que si
si plusieurs connections tombent en panne, il se peux que la communication dans le serveur soit impossible.
Heureusement, le système en double boucle corrige ce défaut, en rajoutant des cannaux de connection secondaire,
afin que les messages puissent toujours être transmit même si les cannaux principaux tombent en panne.\\
Le maillage irrégulier est un graphe assez classique, les éléments sont liés entre eux via des connections, mais
chaque élément n'est pas lié à tout les autres. Ce dernier cas n'arrive que dans le maillage régulier,
qui obtient donc le titre de graphe complet.

\subsubsection{Communication avec connection}
Dans le système de communication avec connection, il faut que les deux machines soient d'accord pour que la
transmission soit faites, comme dans un appel téléphonique, par exemple.\\
Chacune des communications faites ainsi possède donc un circuit par lequel vont passer toutes les données, et ce
circuit disparaît à la fermeture de la communication. Donc, deux communication entre deux même machines ont peu de
chances d'avoir exactement le même circuit, même si la seconde communication est lancé très peu de temps après
que la première soit terminée.

\subsubsection{Communication sans connection}
Dans la communication sans connexion, la machine émetrisse envoie le message sans s'assurer que la machine
destinatrice, ou que les machines intermédiaires, ne soient actifs et capable de reçevoir et transmettre le message.
Un bon exemple de communication sans connection est le service postale.\\
Étant donné qu'il n'y a pas de connection établie par les machine, il ne peux pas y avoir de circuit tracé en
avance, et les paquets n'arriveront donc probablement pas dans l'ordre dans lequel ils ont été envoyés.\\
Cependant, l'ordre d'arrivé des paquets n'est pas aléatoire, il sera, le plus souvent, déterminé par des
algorithmes de parcours de graphe, tel que Dijkstra.

\subsubsection{Commutation de circuit}
Si un circuit n'est pas beaucoup utilisé, il peux être réalloué, au moins partiellement, afin que le réseau n'ai pas
une zone qui est inutilisable par le reste du réseau, mais dans laquelle aucun paquet ne voyage.\\
Les noeuds d'un réseau ont une capacité de mémoire, afin de pouvoir stocker provisoirement le message, et de pouvoir
le renvoyer si il n'a pas été transmis sans echec.

\subsubsection{Commutation de paquets}
Plutôt que d'envoyer les messages d'un coup, ils sont découpés en paquets plus petits, puis envoyés.\\
Le destinataire aura donc comme travail de reconstituer le message, car les paquets n'arriveront pas forcément dans 
l'ordre envoyé.\\
Afin de l'assister dans cette tache, le dernier paquet contiendra un bit indiquant qu'il correspond à la fin du
message.\\

\subsubsection{Commutation de cellules}
Une cellule est un paquet de précisément 53 octets. Il est avantageux de transmettre de si petits paquets,
car leur taille leur permet d'être transmis avec des risques de perte moindre, et sont plus rapide à envoyer.

\subsubsection{Normes}
Les normes sont mis en place par des organismes de normalisation, afin que tout les réseaux puissent facilement
communiquer entre eux. Des exemples d'organisation de normalisation sont:
\begin{itemize}
	\item L'ISO  (International Standard Organisation)
	\item L'ITU  (International Telecomumnication Union)
	\item L'IEEE (Institute of Electrical and Electronic Engineers)
\end{itemize}

\subsection{Modèle OSI}
Le modèle OSI(Open Systems Interconnection) est un système en couches, ce qui permet une séparation des parties du
réseau.\\
Cette séparation permet, entre autres, de changer une partie du réseau, ou couche, sans devoir toucher les autres.

\subsubsection{Encapsulation des données}
Dans le modèle OSI, un message qui part de la couche n de la machine A, pour aller vers la couche n de la machine B
doit passer par toutes les couches intermédiares des deux machines.\\
Chaque couche intermédiare de la machine A va donc rajouter un en-tête au message, afin que la couche intermédiare
correspondante de la machine B sache quoi faire du message.\\
L'acte de rajouter cette en-tête s'appelle l'encapsulation des données.

\subsubsection{Couche physique}
La couche physique permet la conncetion physique entre les machines communicantes, c'est à dire l'envoie du message,
en bits, entre les machines. Il est impératif que son message ne soit pas altérer, pour éviter les erreurs de
communications.\\
Au niveau de cette couche, la communication peux se faire de multiples façons:
\begin{itemize}
	\item une communication en simplex
	\item une communication en half-duplex
	\item une communication en full-duplex
\end{itemize}

La communication en simplex est une communication unilatérale, ou la couche transmet son message, et ne recevra
jamais de réponse.\\
Dans la communication en half-duplex, les deux couches discutent, mais une couche ne peux envoyer un
message que si aucun autre message n'est entrain d'être envoyé. Elles communiquent donc en alternance.\\
Avec la communication en full-duplex, les deux couches discutent, et s'envoient des messages quand elles veulent.
Cela correspond à une discussion entre deux personnes ou aucun des deux participants n'attend que l'autre ai fini
avant de parler.

Le débit est exprimé en Baud, et un Baud correspond à un top d'horloge par seconde.\\
Le débits en bits/s est équivalent au débit en Baud/s.
