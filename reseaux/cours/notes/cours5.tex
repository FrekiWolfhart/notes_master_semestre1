\section{Cours 5}
\subsection{IPv6}
L'IPv6 contient tellement d'adresses, qu'elle rend inutile le NAT.
Les adresses IPv6 ont trois catégories:
\begin{enumerate}
	\item Les adresses unicast, désignant une seule interface.
	\item Les adresses anycast, désignant un ensemble d'interfaces. Ce type n'existe pas en IPv4.
	\item Les adresses multicast, désignant un ensemble d'interfaces.
\end{enumerate}
\textbf{Remarque:} Il n'y a pas d'adresse de broadcast, celle ci étant remplacé par l'adresse multicast.
Une interface peux avoir plusieurs adresses IPv6, voir les trois en même temps, si on veut.

Lorsque l'adresse IPv6 est écrite en héxadécimale, on peut omettre les 0 non significatifs,
i.e ceux avant le premier caractère significatif.//
Par exemple, on peux écrire 0025 25 dans une adresse sans que cela ne pose problème.
Dans une adresse IPv6, on peux remplacer une longue suite de 0 par ::, mais ce symbole ne doit apparaître
qu'une seule fois pour éviter l'ambiguité. Par exemple, 0:0:0:0:0:0:0:1 devient ::1.

On peux représenter une adresse IPv4 en IPv6 en ajoutant ::ffff: devant.
Par exepmle, 127.0.0.1 en IPv4 devient ::ffff:127.0.0.1 en IPv6.

Pour éviter une ambiguité avec les ::, l'adresse IPv6 est mise entre [] dans une url.
Par exemple, http://[100:37:29:145::15]:80 indique qu'on se connecte à l'adresse IPv6 100:37:29:145::15, et au port 80.

\subsubsection{Adresses unicast}
Les adresses unicast ont différent types:
\begin{itemize}
	\item globales (portée globale: internet)
	\item de lien local (portée au niveau du lien: switch ou hub, préfixe: fe80::/64)
	\item de site local, mais c'est un type déprécié
	\item locales uniques (portée au site, préfixe: fc00::/7, dédiées à des communications locales)
\end{itemize}

\subsubsection{Adresses multicast}
Les adresse multicast ont un champ porté, qui définit leur portée ainsi:
\begin{itemize}
	\item 1: interface locale
	\item 2: lien local
	\item 5: site local
	\item 8: organisation locale
	\item E: global
\end{itemize}

\subsubsection{Adresses anycast}
Les adresses anycast sont prises du domaine des adresses unicast, et sont utilisées par les routeurs.

\subsubsection{Les états et durée de vie}
Chaque adresse IPv6 a une durée de vie qui détermine sa validité. Elle peut être infinie.
Une fois la durée de vie finie, l'adresse devient expirée, et est maintenant assignable à une autre interface.
Avant d'être expirée, elle passe dans un état déprécié, ce qui lui permet d'indiquer qu'il faut lui assigner
une nouvelle adresse avant la date fatidique.

Afin de garantir l'unicité d'une adresse IPv6, on utilise la procédure DAD(Duplicate Address Detection).

\subsubsection{Datagramme}
En IPv6, le pacquet n'est pas fragmenté par les routeurs intermédiares. Si sa taille est supérieure à la capacité du
réseau, le routeur le supprime et envoie une alerte "Package Too Big" à la source.

Les entêtes créés par les couches intermédiares peuvent être ignorées par les routeurs intermédiares.
