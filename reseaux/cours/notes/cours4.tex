\section{Cours 4}
Il y a des adresses IP interdites pour les machines. La page 91 du poly de cours contient la liste d'adresses qu'on
ne peux pas assigner à une machine, ainsi que les raisons.

La machine peux envoyer un message vers son local host 127.0.0.1, ce qui peux servir à communiquer entre plusieurs
processus et plusieurs programmes qui tournent sur la machine.

Dans un réseau, une adresse IP privée sert à reconnaître la machine à l'interieur du réseau.

\subsection{Le système CIDR}
Le système CIDR(Classless Inter Domain Routing) est apparu en 1993, et a des adresses de la forme A.B.C.D/M, avec
M étant le masque de réseau.
Ce système a été crée afin de régler le problème de pénuries d'adresses de classe B.

Un switch, comparé à un hub, va retenir les adresses MAC des machines, et ne va plus diffuser le message sur tout le
réseau. Et les tables peuvent être mise à jour si les adresses changent. Cela diminue la surcharge du réseau, car le
message est ciblé sur une machine. Cela réduit aussi les risques de collisions.

\subsection{Réseau ad-hoc}
On peux créer un réseau ad-hoc entre machines, qui sert à relier des machines sans infrastructure.
