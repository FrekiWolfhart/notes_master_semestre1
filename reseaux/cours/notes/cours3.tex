\section{Cours 3}
\subsection{Le modèle TCP/IP}
Le modèle OSi étant un modèle théorique, le modèle TCP/IP ne suit pas son architecture à la lettre/
Les principales différences sont:
\begin{itemize}
	\item Les couches physique et liaisons de données fusionnées dans TCP/IP.
	\item Les trois couches applicatives fusionnées dans TCP/IP.
\end{itemize}
\subsubsection{La couche de liens d'internet}
Elle se charge de transmettre des datagrammes IP, et de gérer les requêtes ARP.

La carte réseau appartient à la couche physique, et est nommée par l'adresse MAC(Medium Access Control).
Elle est suposée unique par machine. Elle tient sur 6 octets, c'est à dire 48 bits, et sont écrites sous format
hexadécimal.

La couche IP ne peux pas renseigner l'adresse MAC d'un pc. Elle ne peux que renseigner son adresse IP.
La requête ARP sert à demander à une machine sur le réseau son adresse MAC, afin de pouvoir lui envoyer un message.
Il existe aussi le protocole RARP, qui permet de déterminer l'adresse IP d'une machine, en ayant seulement son
adresse MAC.

Le cache ARP sert à stocker la requête ARP, afin de ne pas avoir à refaire la requête à chaque fois.
Le cache à une durée de vie limitée, car les adresses IP et MAC peuvent changer.
Sous linux, la commande arp permet, entre autre, de consulter le cache ARP de la machine.

\subsubsection{Fragmentation}
La fragmentation permet de transmettre des datagrammes trop gros pour le réseau.
