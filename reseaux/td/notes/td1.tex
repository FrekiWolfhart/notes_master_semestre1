\section{TD1}

\subsection{Exercice 1}
\subsubsection{Question 1}
Des connections qui permettent le dialogue bidirectionnel à l'alternat sont connues comme des connections \textbf{Half duplex}.
\subsubsection{Question 2}
Une transmission en bande de base correspond à une transmission en \textbf{numérique}.
\subsubsection{Question 3}
L'interconnexion de machine par l'intermédiare d'un Hub correspond à une topologie en \textbf{bus}.
\subsubsection{Question 4}
Si un message de 60 octets part de la couche réseau de la machine A, alors la couche réseau de la machine B reçevra un message de \textbf{60 octets}.

\subsection{Exercice 2}
\subsubsection{Question 1}
L'utilisation d'un protocole en couches permet de:
\begin{itemize}
	\item changer une couche sans impacter les autres (maintenance)
	\item assurer l'interopérabilité entre les machines
\end{itemize}
et à comme principal défault que le message envoyé puisse devenir très lourd du aux en-tête placées par les couches intermédiares, ou au padding.
\subsubsection{Question 2}
Deux standards permettant l'interopérabilité:
\begin{itemize}
	\item chargeur de smartphone à induction
	\item rails de train
\end{itemize}
Deux standards ne permettant pas l'interopérabilité:
\begin{itemize}
	\item chargeurs de smartphone filiaire
	\item connectique vidéo
\end{itemize}
\subsubsection{Question 3}
La fibre optique à un débit élevé, et une latence élevée, surtout sur des grandes distances.
La wifi, par contre, a un débit faible et une latence faible.
\subsubsection{Question 4}
Non, ils ne sont pas identiques. Le flux d'octets envoie des messages plus légers, mais sans doute désordonnées,
alors que le message est une entité unique et indivisible à travers tout le réseau, et est donc plus lourd.
\subsubsection{Question 5}
Il y aura peu d'impacte, voir aucun, car le modèle en couche permet de changer une couche sans toucher les autres.
\subsubsection{Question 6}
\begin{tabular}{|c|c|c|c|}
\hline
 & étoile & anneau & maillage complet \\
\hline
court & 2 & 1 & 1 \\
\hline
moyen & 2 & $\frac{n}{4}$ & 1 \\
\hline
long & 2 & $\frac{n}{2}$ & 1 \\
\hline
\end{tabular}

Le mailage complet est plus rapide, mais est plus cher à implémenter.
Dans le cadre de cet exercice, l'anneau bidirectionnel priorisait le chemin le plus court.
