\section{TD 2}
\subsection{Exercice 3}
Le code C est le code regroupant les 4 mots utilisés dans cet exercice.
\subsubsection{Question 1}
La distance de Hamming entre deux mots est le nombre de bits différents entre les mots.\\
La distance de Hamming d'un code correspond à la distance de hamming minimum entre les mots de ce code, sans prendre
en compte la distnce entre un mot et lui même. Dans cet exercice, c'est 5.
\subsubsection{Question 2}
Un code de Hamming de distance d peux détecter d-1 erreurs.
De plus, un code de Hamming de distance $d=2k+1$ peux corriger k erreurs.
Par conséquent, concernant le code C, il peux détecter 4 erreurs(d-1), et en corriger 2(2k+1).
\subsubsection{Question 3}
Le mot résultat est 1110000000. On cherche le mot initial, et, pour cela le mot le plus proche dans le code.
On va donc corriger le mot en le remplaçant par le mot le plus proche dans C, qui est  1111100000.

\subsection{Exercice 4}
Dans cet exercice, on utilise un code polynomial généré par le polynôme $G(x)=x^4+x+1$, et le mot m = 110110.\\

Un code à redondance cyclique, noté CRC, a pour principe d'ajouter plusieurs bits de contrôle, dont les valeurs sont
des combinaisons du mot initial m.
Pour faire cela, on utilise un polynôme générateur, souvent noté par G.
Le degré de G correspond au nombre de bits de contrôle ajoutés. Donc, dans cet exercice, 4 bits de contrôle seront
rajoutés.
Afin d'obtenir les bits de contrôle, on prends le message m, celui qui doit reçevoir les bits de contrôle, puis on
essaye de l'écrire sous forme de ploynôme.
Dans cet exercice, cela donne $m(x)=x^5+x^4+x^2+x$. Ensuite, on multiplie le degré de G à ce polynôme.
Cela donne: $m(x)=x^4(x^5+x^4+x^2+x)=x^9+x^8+x^6+x^5$.
Ensuite, on fait une division euclidienne entre ce polynôme et G(x).
Pour faire cela, on élimine le plus grand facteur de m(x) à chaque nouvelle division. On continue jusqu'à ce que
le plus grand facteur de m(x) soit inférieur à celui de G(x).
Dans cet exercice, ça donne: $\frac{m(x)}{G(x)}=x^5+x^4+x+1$, et à comme reste $r(x)=x^2+1$.
Le reste r(x) servira à calculer les bits de contrôle. Étant donné qu'il faut 4 bits de contrôle, on convertit cette
équation en binaire.
Cela donne: $r(x)=x^2+1=0x^3+1x^2+0x^1+1x^0$. Par conséquent, les bits de contrôle de m seront 0101.
Donc, le mot écrit avec les bits de contrôle sera m=1101100101.
Les bits de contrôle sont aussi appelés bits de redondance car ils son rajoutés à l'information initiale.
