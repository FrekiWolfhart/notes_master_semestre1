\section{TD 3}
Les adresses IP contiennent des 0 inutiles afin de bien tout aligner en colonne.
\subsection{Exercice 1}
\begin{itemize}
	\item 141.115.004.005: adresse de classe B, id réseau: 141.115.0.0, id machine : 0.0.4.5
	\item 006.324.012.013: adresse impossible car les chiffres vont de 0 à 255
	\item 001.001.001.002: adresse de classe A, id réseau:1.0.0.0, id machine:0.1.1.2
	\item 141.115.000.000: adresse réseau de classe B, interdite à une machine
	\item 126.024.015.002: adresse de classe A, id réseau: 126.0.0.0, id machine:0.24.15.2
	\item 210.255.255.000: adresse de réseau de classe C, interdite à une machine
\end{itemize}

Adresses interdites pour les machine:
\begin{itemize}
	\item adresses réseau (tout les bits à 0 sauf ceux d'id réseau)
	\item tout les bits à 0, ou tout les bits à 1
	\item tout les octets hors de l'id réseau à 1
	\item adresse de rebouclage (127.x.y.z)
\end{itemize}

\subsection{Exercice 2}
\subsubsection{Question 1}
La partie identifiant le réseau est 155.102.0.0 car l'adresse est de classe B.
\subsubsection{Question 2}
\begin{itemize}
	\item 155.102.015.000 n'appartient pas au réseau
	\item 155.102.015.064 est l'adresse du réseau
	\item 155.102.015.065 appartient au réseau
	\item 155.102.015.200 n'appartient pas au réseau
\end{itemize}

\subsection{Exercice 3}
\subsubsection{Question 1}
Le réseau définie par l'adresse 214.123.155.0/24 est un réseau de classe C.
\subsubsection{Question 2}
Afin d'avoir une dizaine de sous réseau, il faut réserver les 4 bits les plus à gauche de
l'id machine. Il serviront à identifier le sous-réseau auquel appartient la machine.
Le nouveau masque sera donc (24+4=28): 214.123.155.0/28.
\subsubsection{Question 3}
Chaque sous-réseau pourra recevoir 16 adresses IP, dont 14 pour les machines.
\subsubsection{Question 4}
Pour le troisième sous-réseau utilisable, on a:
\begin{itemize}
	\item adresse réseau: 214.123.155.48.
	\item adresse de broadcast: 214.123.155.63.
\end{itemize}


