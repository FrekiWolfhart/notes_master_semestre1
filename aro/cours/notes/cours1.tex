\section{Cours 1}
\subsection{La programmation linéaire}
Exemple de problème linéaire:
maximiser $x_1 + x_2$ avec:
\begin{itemize}
    \item $x_2 - x_1 \leq 1$
    \item $x_1 + 6 x_2 \leq 15$
    \item $4x_1 - x_2 \leq 10$
    \item $x_1 \geq 0$, et$x_2 \geq 0$
\end{itemize}

Si on dessine ce problème sur un plan, on observe une
zone du plan correspondant à ces contraintes, et la
solution optimale avec ces contraintes est 5, avec
$x_1 = 3$ et $x_2 = 2$.

\subsection{La méthode simplex}
Il s'agit d'aller de solution de base en solution
de base.\\
Une solution de base est l'ensemble des sommets du
polyhèdre, et le polyhèdre est l'ensemble des solutions
admissibles.\\
La méthode simplex est donc une méthode de voyage entre 
les sommets du polyhèdre solution.

Forme générale du problème linéaire:
maximiser $c_1x_1 + c_2x_2 + ... + c_nx_n$ avec:
\begin{itemize}
    \item $a_{11}x_1 + a_{12}x_2 + ... + a_{1n}x_n \leq b_1$
    \item $a_{m1}x_1 + a_{m2}x_2 + ... + a_{mn}x_n \leq b_n$
    \item $x_1, ..., x_n \geq 0$
\end{itemize}

Chaque problème de maximisation a un problème de
minimisation associé.

La programmation linéaire date de Fourrier, alors que
la méthode suplex a été inventée dans les années 50 par 
Dantzig.

L'algorithme de la méthode simplex est exponentiel dans
le pire des cas, et polynomial en moyenne.

L'algorithme de Khachian, créé dans les années 70 est un
algorithme polynomial un peu moins efficace que la
méthode suplex.

Prenons l'exemple de la diète. Notre corps à besoin de:
\begin{itemize}
    \item Vitamine A : 0,5mg
    \item Vitamine C : 15mg
    \item Fibres : 4g
\end{itemize}

Chaque nourriture à un prix, et notre corps à des
demandes nutritionnelles.

\begin{tabular}{|c|c|c|c|c|}
\hline
Nom nutriment & Carottes($x_1$) & Choux($x_2$) & Cornichons($x_3$) & Demandes \\
\hline
Vitamine A & 35mg & 0,5mg & 0,5mg & 0,5mg\\
\hline
Vitamine C & 60mg & 30mg & 10mg & 15mg\\
\hline
Fibres & 30g & 20g & 10g & 4g\\
\hline
prix & 0,75 euro & 0,5 euro & 0,15 euro & \\
\hline
\end{tabular}

Cette exemple demandera donc de \textbf{MINIMISER} la
quantité d'aliments à prendre, afin de minimiser le
coût, et d'avoir les apports minimaux requis.

Exemple de la méthode suplex:\\

\begin{tabular}{|c|c|c|c|c|c|}
\hline
Objet & Cendrier($x_1$) & Bol($x_2$) & Cruche($x_3$) & Vase($x_4$) & maximum \\
\hline
Moulage & 2 & 4 & 5 & 7 & 42 \\
\hline
Cuisson & 1 & 1 & 2 & 2 & 17 \\
\hline
Peinture & 1 & 2 & 3 & 3 & 24 \\
\hline
Bénéfice & 7 & 9 & 18 & 17 &  \\
\hline
\end{tabular}

On veut donc, dans ce problème, maximiser $7x_1 + 9x_2 + 18x_3 + 17 x_4$ avec:
\begin{itemize}
    \item $2x_1 + 4x_2 + 5x_3 + 7x_4 \leq 42$
    \item $1x_1 + 1x_2 + 2x_3 + 2x_4 \leq 17$
    \item $1x_1 + 2x_2 + 3x_3 + 4x_3 \leq 24$
    \item $x_1,x_2, x_3, x_4 \geq 0$
\end{itemize}

Dictionnaire D1:
\begin{itemize}
    \item $x_5 = 42 - 2x_1 - 4x_2 - 5x_3 - 7x_4$
    \item $x_6 = 17 - 1x_1 - 1x_2 - 3x_3 - 2x_4$
    \item $x_7 = 24 - 1x_1 - 2x_2 - 3x_3 - 3x_4$
    \item $trait de division$
    \item $z = 7x_1 + 9x_2 + 18x_3 + 17 x_4$
\end{itemize}

Les variables de base sont les x de 5 à 7, et les
variables hors base sont les x de 1 à 4.

La solution de base admissible consiste à affecter à
toutes les variables hors base la valeur 0, ce qui crée:
\begin{itemize}
    \item $x_5 = 42$
    \item $x_6 = 17$
    \item $x_7 = 24$
    \item $z = 0$
\end{itemize}

La méthode suplex va consister à faire entrer une
variable hors base dans la base, et à faire sortir une
variable de base de la base.

On va donc faire entrer la varible $x_3$, car c'est la
variable qui donne le plus de bénéfices
(technique glouton). La variable qui va sortir sera
celle qui mets le plus de contraintes, donc qui limite
le plus la variable entrante. Il s'agit de $x_7$ dans
ce cas.

L'étape de pivot fait donc entrer $x_3$ et sortir $x_7$.
Donc, on a:
$$3x_3 = 24 - 1x_1 - 2x_2 - 3x_4 - x7$$
Ce qui donne:
$$x_3 = 8 - \frac{1}{3}x_1 - \frac{2}{3}x_2 - x_4 - \frac{1}{3}x_7$$

Le dictionnaire D2 est crée en remplaçant $x_3$ par l'expression en haut sur toutes les lignes de D1. En le
simplifiant, ça donne:
\begin{itemize}
    \item $x_3 = 8 - \frac{1}{3}x_1 - \frac{2}{3}x_2 - x4 - \frac{1}{3}x_7$
    \item $x_5 = 2 - \frac{1}{3}x_1 - \frac{2}{3}x_2 - 2x_4 - \frac{5}{3}x_7$
    \item $x_6 = 1 - \frac{1}{3}x_1 - \frac{1}{3}x_2 - \frac{2}{3}x_7$
    \item $trait de division$
    \item $z = 144 + x_1 - 3x_2 - x_4 - 6x_7$
\end{itemize}

La solution de base admissible dans ce cas est 144 euros
gagnés.

La variable entrante de D2 est $x_1$, car c'est la seule
encore positif dans l'équation avec z. La variable
sortante est $x_6$. Ce qui donne:
$$\frac{1}{3}x_1 = 1 + \frac{1}{3}x_2 - x_6 + \frac{2}{3}x_7$$
Ce qui donne, après simplification:
$$x_1 = 3 + x_2 - 3x_6 + 2x_7$$
On obtient D3 en remplaçant $x_1$ par cette équation
dans tout D2. Après simplification, ça donne:
\begin{itemize}
    \item $x_1 = 3 + x_2 - 3x_6 + 2x_7$
    \item $x_3 = 7 - x_2 - x_4 + x_6 - x_7$
    \item $x_5 = 1 - x_2 - 2x_4 + x_6 + x_7$
    \item $trait de division$
    \item $z = 147 -2x_2 - x_4 - 3x_6 - 4x_7$
\end{itemize}

Étant donné qu'il n'y a plus de variable positive dans
l'équatikn avec z, la solution de base admissible de D3
est la solution optimale. Donc, optimalement, on fera
3 cendriers, 0 bols, 7 cruches et 0 vases, et on fera
un bénéfice de 147 euros.

\subsubsection{Cas spéciaux}
Cas spécial 1:maximum non borné\\
Si le maximum n'est pas borné, alors maximiser retourne $+\infty$.\\
Si il n'y a pas de contrainte sur $x_2$, alors le
maximum va vers l'infini.

Cas spécial 2: Dégénérescence et cyclage\\
La dégénérescence, c'est quand la borne supérieure de la
variable entrante est 0. Dans ce cas, on fait entrer
dans la base cette variable, même si on ne gagne rien.

Le cyclage, c'est quand des étapes dégénèrent et nous
ramènent toujours aux mêmes dictionnaires. Il faudra
appliquer la règle de Blund pour s'en sortir.

La règle de Blund consiste à toujours choisir les
variables entrentantes et sortantes de plus petit indice
, si jamais il y a plus d'une variable possible.

\textbf{Théorème} La méthode suplex avec la règle de
Blund n'a pas de cyclage.

Cas spécial 3: Recherche d'un dictionnaire et
infaillibilité\\
Il faut utiliser la méthode suplex à deux phases, ce qui
permet, via l'introduction d'un programme linéaire
auxiliaire, de gérer le cas où créer la solution de base
admissible du dictionnaire donne des x négatifs.
